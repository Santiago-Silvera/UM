\documentclass{report}
\usepackage[spanish]{babel}
\usepackage{mathtools}
\usepackage{dsfont}
\usepackage{amssymb}
\usepackage{tcolorbox}

\newcommand\defeq{\stackrel{\mathclap{\normalfont\mbox{def}}}{~=~}}
\newcommand\reals{\mathds{R}}

\newtcolorbox[auto counter,number within=chapter]{defbox}[2][]{
colback=red!5!white,colframe=red!75!black,fonttitle=\bfseries,
title=Definición~\thetcbcounter: #2,#1}

\newtcolorbox[auto counter,number within=chapter]{propobox}[1][]{
colback=green!5!white,colframe=green!75!black,fonttitle=\bfseries,
title=Proposición~\thetcbcounter, #1}

\newtcolorbox[auto counter,number within=chapter]{teobox}[2][]{
colback=cyan!5!white,colframe=cyan!75!black,fonttitle=\bfseries,
title=Teorema~\thetcbcounter: #2,#1}


\begin{document}

\chapter{Integrales sobre curvas}
\section{Integrales de campos escalares sobre curvas}
\begin{defbox}{Parametrización}
	Llamamremos $\textbf{parametrización}$ a toda función \(\alpha : I\subseteq\reals\rightarrow\reals^3\), donde \(I\) es un intervalo en \(\reals\).
\end{defbox}

\begin{defbox}{Curva suave}
	Sea \(\mathcal{C}\) un conjunto no vacío de \(\reals^3\). \\
	Diremos que \\
	$\mathcal{C}$ es una \textbf{curva suave} $\iff$ \underline{existe} una parametrización $\alpha : I\subseteq\reals\rightarrow\reals^3$ de clase $C^1$ tal que $\mathcal{C}=\alpha (I)$
\end{defbox}

\begin{defbox}{Inmersión}
	Consideremos una parametrización $\alpha : I \subseteq \reals \rightarrow \reals^3$. \\

	Diremos que $\alpha$ es una \textbf{inmersión} \(\iff
	\left\{ \begin{array}{l}
		1. ~\alpha \text{ es de clase }C^1 \\
		2. ~\alpha '(t) \ne 0 ~~~ \forall t \in I
	\end{array}\right.\)
\end{defbox}

\begin{defbox}{Curva regular}
	Sea \(\mathcal{C}\) un conjunto no vacío de \(\reals^3\). \\
	Diremos que \\
	$\mathcal{C}$ es una \textbf{curva regular} $\iff$ \underline{existe} una inmersión $\alpha : I\subseteq\reals\rightarrow\reals^3$ de clase $C^1$ tal que $\mathcal{C}=\alpha (I)$
\end{defbox}

\begin{defbox}{Curva simple}
	Sea \(\mathcal{C}\) un conjunto no vacío de \(\reals^3\). \\
	Diremos que \\
	$\mathcal{C}$ es una \textbf{curva simple} $\iff$ \underline{existe} una inmersión $\alpha : I\subseteq\reals\rightarrow\reals^3$ de clase $C^1$ tal que $\mathcal{C}=\alpha (I)$ y además $\alpha$ es \underline{inyectiva en el interior de I}
\end{defbox}

\begin{defbox}{Curva cerrada}
	Sea \(\mathcal{C}\) un conjunto no vacío de \(\reals^3\). \\
	Diremos que \\
	$\mathcal{C}$ es una \textbf{curva cerrada} $\iff$ \underline{existe} una inmersión $\alpha : [a,b] \subseteq\reals\rightarrow\reals^3$ de clase $C^1$ tal que $\mathcal{C}=\alpha (I)$ y se cunple que $\alpha (a) = \alpha (b)$
\end{defbox}

\begin{defbox}{Cambio de parámetros}
	Consideramos una función $h: I \rightarrow \reals$, donde $I$ es un intervalo de $\reals$. \\
	Diremos que \\
	h es un \textbf{cambio de parámetros} $\iff
		\left\{ \begin{array}{cc}
			1)~ h\text{ es de clase } C^1 \\
			2)~ h'(t) \ne 0 ~~~ \forall t \in I
		\end{array}\right.
	$
\end{defbox}

\begin{propobox}[label=propo:reparametrizacion]
	Sea $\alpha : I \rightarrow \reals^3$ es una parametrización de clase $C^1$ de una curva $\mathcal{C} \subset \reals^3$ y $h: I \rightarrow \reals^3$ un cambio de parámetros, y sea $J=h(I)$. \\
	Entonces \\
	(1) la función
	\[
		\beta \defeq \alpha \circ h^{-1}:J\rightarrow \reals^n
	\]
	es una parametrización de clase $C^1$ de la curva $\mathcal{C}$, llamada \textbf{reparametrización} de $\alpha$ a través del cambio de parámetros $h$. \\
	(2) Si $\alpha$ es inyectiva entonces $\beta$ también es inyectiva. \\
	(3) Si $\alpha$ es una inmersión entonces $\beta$ también es una inmersión.
\end{propobox}

\begin{propobox}[label=propo:existencia_cambio_param]
	Si $\alpha : I \rightarrow \reals^3$ y $\beta : J \rightarrow \reals^3$ son dos inmersiones inyectivas de una misma curva $\mathcal{C}\subseteq\reals^3$ entonces existe un cambio de parámetros $h: I \rightarrow \reals$ con $h(I)=J$ tal que $\beta = \alpha \circ h^{-1}$
\end{propobox}

\begin{defbox}{Integral de un campo escalar sobre una curva}
	Sea $\mathcal{C}\subset\reals^3$ una curva regular simple y $f: U \subset \reals^3 \rightarrow \reals$ un campo escalar continuo en un abierto $U$ que contiene a $\mathcal{C}$. Si $\alpha : [a,b] \rightarrow \reals^3$ es una inmersión de la curva $\mathcal{C}$ definimos la integral del campo escalar $f$ sobre la curva $\mathcal{C}$ como:
	\begin{equation}
		\boxed{\int_{\mathcal{C}}{fds}\defeq \int_a^b{f(\alpha (t)) ~||\alpha '(t) ||dt}}
		\label{eq:int_c_f}
	\end{equation}
\end{defbox}

\begin{teobox}{Independencia respecto a la parametrización}
	Sea $\mathcal{C} \subset \reals^3$ una curva regular simple y $f: U \subset \reals^3 \rightarrow \reals$ un campo escalar continuo en un abierto $U$ que contiene a $\mathcal{C}$. \\
	Si $\alpha :[a,b] \rightarrow \reals^3$ y $\beta : [c,d] \rightarrow \reals^3$ son dos inmersiones inyectivas de la curva $\mathcal{C}$ entonces se cumple que: \\
	\[
		\int_a^b{f(\alpha (t))||\alpha '(t)||dt} = \int_c^d{f(\beta (s))||\beta '(s)||ds}
	\]
	\tcblower
\end{teobox}
\underline{Demostración} \\
Por la proposición \ref{propo:reparametrizacion} y \ref{propo:existencia_cambio_param} sabemos que $\beta$ es una reparametrización de $\alpha$ a través de un cambio de parámetros $h: [a,b] \rightarrow [c,d]$ esto es: \\
\[
	\alpha = \beta \circ h
\]
Primer caso: Si $h'(t) > 0 ~~~ \forall t \in [a,b]$ la función $h: [a,b] \rightarrow [c,d]$ es creciente y se cumple que
\[
	\left\{ \begin{array}{l}
		h(a) = c \\
		h(b) = d
	\end{array} \right.
\]
Luego \\
\[
	\begin{array}{ll}
		\int_a^b{f(\alpha (t)) ||\alpha '(t)||dt} & = \int_a^b{f(\beta (h(t))) ||[\beta (h(t))]'||dt}    \\
		                                          & = \int_a^b{f(\beta (h(t))) ||\beta '(h(t))h'(t)||dt} \\
		                                          & = \int_a^b{f(\beta (h(t))) ||\beta '(h(t))||h'(t)dt} \\
		                                          & = \int_{h(a)}^{h(b)}{f(\beta (s) ||\beta '(s)||ds}   \\
		                                          & = \int_{c}^{d}{f(\beta (s) ||\beta '(s)||ds}         \\
	\end{array}
\]
Segundo caso: Si $h'(t) < 0 ~~~ \forall t \in [a,b]$ la función $h: [a,b] \rightarrow [c,d]$ es decreciente y se cumple que
\[
	\left\{ \begin{array}{l}
		h(a) = d \\
		h(b) = c
	\end{array} \right.
\]
Luego \\
\[
	\begin{array}{ll}
		\int_a^b{f(\alpha (t)) ||\alpha '(t)||dt} & = \int_a^b{f(\beta (h(t))) ||[\beta (h(t))]'||dt}     \\
		                                          & = \int_a^b{f(\beta (h(t))) ||\beta '(h(t))h'(t)||dt}  \\
		                                          & = -\int_a^b{f(\beta (h(t))) ||\beta '(h(t))||h'(t)dt} \\
		                                          & = -\int_{h(a)}^{h(b)}{f(\beta (s) ||\beta '(s)||ds}   \\
		                                          & = -\int_{d}^{c}{f(\beta (s) ||\beta '(s)||ds}         \\
		                                          & = \int_{c}^{d}{f(\beta (s) ||\beta '(s)||ds}          \\
	\end{array}
\]

\section{Integrales de campos vectoriales sobre curvas}

\begin{defbox}{Espacio de direcciones tangente}
	Sea $\mathcal{C}$ una curva regular y simple en $\reals^3$ y $p\in \mathcal{C}$. \\
	Si $\alpha : I \rightarrow \reals^3$ es una inmersión inyectiva de la curva $\mathcal{C}$ existe un único $t\in I$ tal que
	\[
		p = \alpha (t)
	\]
	Definimos el \textbf{espacio de direcciones tangentes} $\mathcal{C}$ en $p$, y lo indicamos por $T_p(\mathcal{C})$, al subespacio vectorial de $\reals^3$ generado por $\alpha (t)$, esto es
	\[
		T_p(\mathcal{C}) \defeq [\alpha '(t)]
	\]
	Y diremos que \\
	$v \ne 0$ es una \textbf{dirección tangente} a $\mathcal{C}$ en $p \iff v \in T_p(\mathcal{C})$
\end{defbox}

\begin{defbox}{Orientación}
	Sea $\mathcal{C}$ una curva regular y simple en $\reals^3$. \\
	Un campo vectorial $\vec{T}: \mathcal{C} \rightarrow \reals^3$ es una \textbf{orientación} en $\mathcal{C} \iff$ cumple las siguientes propiedades:
	\begin{enumerate}
		\item $\vec{T}$ es continuo \\
		\item $\vec{T}$ es un versor (esto es $||\vec{T}||=1$) \\
		\item $\vec{T}(p)$ es una dirección tangente a $\mathcal{C}$ en $p$ (esto es $\vec{T}(p)\in T_p(\mathcal{C})$)
	\end{enumerate}
	Una \textbf{curva orientada} es una curva en la cual hemos elegido una orientación.
\end{defbox}

\begin{defbox}{Integral de un campo vectorial sobre una curva}
	Sea $\mathcal{C} \subset \reals^3$ una curva regular simple orientada por el campo continuo $\vec{T}$ de versores tangentes. \\
	Consideremos un campo vectorial $\vec{X}: U \subseteq \reals^3 \rightarrow \reals^3$ continuo en un abierto $U$ que contiene a $\mathcal{C}$. Definimos la integral del campo vectorial $\vec{X}$ a lo largo de la curva $\mathcal{C}$ como:
	\[
		\int_{\mathcal{C}}{\vec{X}ds} \defeq \int_{\mathcal{C}}{\vec{X}\cdot\vec{T}ds}
	\]
\end{defbox}

% Formula de calculo
\begin{propobox}
	Sea $\mathcal{C} \subset \reals^3$ una curva regular simple orientada por el campo continuo $\vec{T}$ de versores tangentes y $\vec{X}:U\subseteq\reals^3 \rightarrow\reals^3$ un campo vectorial continuo en un abierto $U$ que contiene a $\mathcal{C}$. \\
	Si $\alpha :[a,b] \rightarrow \reals^3$ es una inmersión de la curva $\mathcal{C}$ entonces
	\begin{equation}
		\boxed{
		\int_{\mathcal{C}}{\vec{X}ds} = \pm \int_a^b{\vec{X}(\alpha (t))\cdot \alpha '(t)dt}
		}
		\label{eq:int_c_vec}
	\end{equation}
\end{propobox}
\underline{Demostración} \\
\[
	\begin{array}{ll}
		\int_{\mathcal{C}}{\vec{X}ds} & \defeq \int_{\mathcal{C}}{\vec{X}\cdot\vec{T}ds}                                                \\
		                              & = \int_a^b{(\vec{X}(\alpha (t))\cdot \vec{T}(\alpha (t)))||\alpha '(t))||dt}                    \\
		                              & =\pm \int_a^b{(\vec{X}(\alpha (t))\cdot \frac{\alpha '(t)}{||\alpha '(t)||})||\alpha '(t))||dt} \\
		                              & = \pm \int_a^b{\vec{X}(\alpha (t))\cdot \alpha '(t)dt}
	\end{array}
\]

\chapter{Integrales sobre superficies}
\section{Integrales de campos escalares sobre superficies}
\begin{defbox}{Parametrización}
	Llamaremos \textbf{parametriación} a toad función $\varphi : D \subseteq \reals^2 \rightarrow \reals^3$, donde $D$ es un conjunto abierto y conexo en $\reals^2$.
\end{defbox}

\begin{defbox}{Superficie suave}
	Sea $\mathcal{S}$ un conjunto no vacío de $\reals^3$. \\
	Diremos que \\
	$\mathcal{S}$ es una \textbf{superficie suave} $\iff$ \underline{existe} una parametrización \\ $\varphi : D \subseteq \reals^2 \rightarrow \reals^3$ de clase $C^1$ tal que $\mathcal{S}=\varphi (D)$ siendo $D$ un conjunto conexo de $\reals^2$.
\end{defbox}

\begin{defbox}{Inmersión}
	Consideremos una parametrización $\varphi : D \subseteq \reals^2 \rightarrow \reals^3$. \\
	Diremos que\\
	$\varphi$ es una \textbf{inmersión}
	$\iff \left\{\begin{array}{l}
			1. ~\varphi \text{ es de clase } C^1 \\
			2. ~\varphi_u(u,v)\times \varphi_v(u,v) \ne 0
		\end{array}\right.$
\end{defbox}

\begin{defbox}{Superficie regular}
	Sea $\mathcal{S}$ un conjunto no vacío de $\reals^3$. \\
	Diremos que \\
	$\mathcal{S}$ es una \textbf{superficie regular} $\iff$ \underline{existe} una inmersión \\ $\varphi : D \subseteq \reals^2 \rightarrow \reals^3$ tal que $\mathcal{S} = \varphi (D)$
\end{defbox}

\begin{defbox}{Superficie simple}
	Sea $\mathcal{S}$ un conjunto no vacío de $\reals^3$. \\
	Diremos que \\
	$\mathcal{S}$ es una \textbf{superficie simple} $\iff$ \underline{existe} una parametrización \\ $\varphi : D \subseteq \reals^2 \rightarrow \reals^3$ de clase $C^1$ tal que $\mathcal{S}=\varphi (D)$ con $\varphi$ \textbf{inyectiva en $D$}.
\end{defbox}

\begin{defbox}{Integral de un campo escalar sobre una superficie}
	Sea $\mathcal{S} \subset \reals^3$ una superficie regular y simple y $f : U \subset \reals^3 \rightarrow \reals$ un campo escalar continuo en un abierto $U$ que contiene a $\mathcal{S}$. Si $\varphi : D \subseteq \reals^2 \rightarrow \reals^3$ es una inmersión inyectiva que cuber a $\mathcal{S}$ definimos la integral del campo escalar $f$ sobre la superficie $\mathcal{S}$ como:
	\begin{equation}
		\boxed{
			\iint_{\mathcal{S}}{fdS} \defeq \iint_{D}{f(\varphi (u, v))||\varphi_u (u,v) \times \varphi_v (u,v)||dudv}
		}
		\label{eq:int_s_f}
	\end{equation}
\end{defbox}

% Lema de algebra v_1 x v_2 = det(a) (w_1 x w_2)
\begin{propobox}[label=propo:lema_alg_lin]
	Sean $v_1, v_2, w_1, w_2$ vectores en $\reals^3$ y $A$ una matriz real $2\times 2$ y las matrices $V$ y $W$ siendo conformadas por los vectores $v_1, v_2$ y $w_1, w_2$ puestos como columnas respectivamente, tales que \\
	\[
		V = WA
	\]
	entonces
	\[
		v_1 \times v_2 = det(A) (w_1 \times w_2)
	\]
\end{propobox}

% unas cosas necias con vectoriales de las derivadas del phi
\begin{propobox}
	Si $\varphi$ es una reparametrización de $\psi$ a través del cambio de parámetros $h$ entonces
	\[
		\varphi_u (u,v) \times \varphi_v (u,v) = det(\mathds{J}h(u,v)) (\psi_s (s,t) \times \psi_t (s,t))
	\]
	donde $(s,t) = h(u,v)$.
\end{propobox}
\underline{Demostración} \\
Tenemos que \\
\[
	\varphi = \psi \circ h
\]
y por la regla de la cadena
\[
	\mathds{J}\varphi (u,v) = \mathds{J}\psi (h (u,v) ) \cdot \mathds{J}h(u,v)
\]
\[
	\iff \mathds{J}\varphi (u,v) = \mathds{J}\psi (s,t) \cdot \mathds{J}h(u,v)
\]
Aplicando la proposición \ref{propo:lema_alg_lin} conluimos \\
\[
	\varphi_u (u,v) \times \varphi_v (u,v) = det(\mathds{J}h(u,v)) (\psi_s (s,t) \times \psi_t (s,t))
\]

\begin{teobox}{Independencia respecto a la parametrización}
	Sea $\mathcal{S} \subset \reals^3$ una superficie regular simple y $f: U \subseteq \reals^3 \rightarrow \reals$ un campo escalar continuo en un abierto $U$ que contiene a $\mathcal{S}$. \\
	Si $\varphi : D \subset \reals^2 \rightarrow \reals^3$ y $\psi : E \subset \reals^2 \rightarrow \reals^3$ son dos inmersiones inyectivas de la superficie $\mathcal{S}$ entonces se cumple \\
	\[
		\iint_D{f(\varphi (u,v) ) ||\varphi_u (u,v) \times \varphi_v (u,v)||~dudv}
	\]
	\[ = \]
	\[
		\iint_E{f(\psi (s,t) ) ||\psi_s (s,t) \times \psi_t (s,t)||~dsdt}
	\]
\end{teobox}
\underline{Demostración} \\
Tenemos que $\psi$ es una reparametrización de $\varphi$ a través de un cambio de parámetros $h: D \rightarrow E$, esto es:
\[
	\psi = \varphi \circ h
\]
Aplicando el cambio de variables (en integral doble)
\[
	\left\{\begin{array}{l}
		(s,t) = h(u,v) \\
		dsdt = |det(\mathds{J}h(u,v))|dudv
	\end{array}\right.
\]
se tiene que \\
\[
	\iint_E{f(\psi (s,t) ) ||\psi_s (s,t) \times \psi_t (s,t)|| dsdt} =
	\iint_E{f(\psi (h(u,v)) ||\psi_s h((u,v)) \times \psi_t (h(u,v))|||det(\mathds{J}h(u,v))| dudv}
\]
pero como $\varphi = \psi \circ h$
\[
	f(\psi (h (u,v) ) ) = f(\varphi (u,v) )
\]
y utilizando la proposición \ref{propo:lema_alg_lin} se deduce que
\[
	||\psi_s (h(u,v)) \times \psi_t (h(u,v))|| = ||\varphi_u (u,v) \times \varphi_v (u,v)||
\]
sustituyendo obtenemos
\[
	\iint_D{f(\varphi (u,v) ) ||\varphi_u (u,v) \times \varphi_v (u,v)||~dudv} =
	\iint_E{f(\psi (s,t) ) ||\psi_s (s,t) \times \psi_t (s,t)||~dsdt}
\]
\section{Integrales de campos vectoriales sobre superficies}

\begin{defbox}{Espacio de direcciones normales}
\end{defbox}

\begin{defbox}{Orientación}
\end{defbox}

\begin{defbox}{Integral de un campo vectorial sobre una superficie}
\end{defbox}

% Formula de calculo usando parametrizaciones
\begin{propobox}
\end{propobox}

\chapter{Teoremas clásicos del cálulo vectorial}

% semejante apampanado el que se puso a demostrar esto
\begin{teobox}{de la curva de Jordan}

	% ni va a ir la demostracion de esta cosa
\end{teobox}


\begin{teobox}{Green}

	\tcblower
	\underline{Demostración} \\
\end{teobox}

% mi gran y querido
\begin{defbox}{Operador de Hamilton}
\end{defbox}

\begin{defbox}{Rotor}
\end{defbox}

\begin{teobox}{Stokes}

	\tcblower
	\underline{Demostración} \\
\end{teobox}

\begin{defbox}{Divergencia}
\end{defbox}

\begin{teobox}{Gauss}

	\tcblower
	\underline{Demostración} \\
\end{teobox}

\end{document}

\documentclass{report}
\usepackage[spanish]{babel}
\usepackage{amsmath,amsthm,mathtools}
\usepackage{dsfont}
\usepackage{amssymb}
\usepackage{tcolorbox}

\newcommand\defeq{\stackrel{\mathclap{\normalfont\mbox{def}}}{~=~}}
\newcommand\reals{\mathds{R}}

\newtcolorbox[auto counter,number within=chapter]{defbox}[2][]{
colback=red!5!white,colframe=red!75!black,fonttitle=\bfseries,
title=Definición~\thetcbcounter: #2,#1}

\newtcolorbox[auto counter,number within=chapter]{propobox}[1][]{
colback=green!5!white,colframe=green!75!black,fonttitle=\bfseries,
title=Proposición~\thetcbcounter, #1}

\newtcolorbox[auto counter,number within=chapter]{teobox}[2][]{
colback=cyan!5!white,colframe=cyan!75!black,fonttitle=\bfseries,
title=Teorema~\thetcbcounter: #2,#1}


\begin{document}

\chapter{Integrales sobre curvas}
\section{Integrales de campos escalares sobre curvas}
\begin{defbox}{Parametrización}
	Llamamremos $\textbf{parametrización}$ a toda función \(\alpha : I\subseteq\reals\rightarrow\reals^3\), donde \(I\) es un intervalo en \(\reals\).
\end{defbox}

\begin{defbox}{Curva suave}
	Sea \(\mathcal{C}\) un conjunto no vacío de \(\reals^3\). \\
	Diremos que \\
	$\mathcal{C}$ es una \textbf{curva suave} $\iff$ \underline{existe} una parametrización $\alpha : I\subseteq\reals\rightarrow\reals^3$ de clase $C^1$ tal que $\mathcal{C}=\alpha (I)$
\end{defbox}

\begin{defbox}{Inmersión}
	Consideremos una parametrización $\alpha : I \subseteq \reals \rightarrow \reals^3$. \\

	Diremos que $\alpha$ es una \textbf{inmersión} \(\iff
	\left\{ \begin{array}{l}
		1. ~\alpha \text{ es de clase }C^1 \\
		2. ~\alpha '(t) \ne 0 ~~~ \forall t \in I
	\end{array}\right.\)
\end{defbox}

\begin{defbox}{Curva regular}
	Sea \(\mathcal{C}\) un conjunto no vacío de \(\reals^3\). \\
	Diremos que \\
	$\mathcal{C}$ es una \textbf{curva regular} $\iff$ \underline{existe} una inmersión $\alpha : I\subseteq\reals\rightarrow\reals^3$ de clase $C^1$ tal que $\mathcal{C}=\alpha (I)$
\end{defbox}

\begin{defbox}{Curva simple}
	Sea \(\mathcal{C}\) un conjunto no vacío de \(\reals^3\). \\
	Diremos que \\
	$\mathcal{C}$ es una \textbf{curva simple} $\iff$ \underline{existe} una inmersión $\alpha : I\subseteq\reals\rightarrow\reals^3$ de clase $C^1$ tal que $\mathcal{C}=\alpha (I)$ y además $\alpha$ es \underline{inyectiva en el interior de I}
\end{defbox}

\begin{defbox}{Curva cerrada}
	Sea \(\mathcal{C}\) un conjunto no vacío de \(\reals^3\). \\
	Diremos que \\
	$\mathcal{C}$ es una \textbf{curva cerrada} $\iff$ \underline{existe} una inmersión $\alpha : [a,b] \subseteq\reals\rightarrow\reals^3$ de clase $C^1$ tal que $\mathcal{C}=\alpha (I)$ y se cunple que $\alpha (a) = \alpha (b)$
\end{defbox}

\begin{defbox}{Cambio de parámetros}
	Consideramos una función $h: I \rightarrow \reals$, donde $I$ es un intervalo de $\reals$. \\
	Diremos que \\
	h es un \textbf{cambio de parámetros} $\iff
		\left\{ \begin{array}{cc}
			1)~ h\text{ es de clase } C^1 \\
			2)~ h'(t) \ne 0 ~~~ \forall t \in I
		\end{array}\right.
	$
\end{defbox}

\begin{propobox}[label=propo:reparametrizacion]
	Sea $\alpha : I \rightarrow \reals^3$ es una parametrización de clase $C^1$ de una curva $\mathcal{C} \subset \reals^3$ y $h: I \rightarrow \reals^3$ un cambio de parámetros, y sea $J=h(I)$. \\
	Entonces \\
	(1) la función
	\[
		\beta \defeq \alpha \circ h^{-1}:J\rightarrow \reals^n
	\]
	es una parametrización de clase $C^1$ de la curva $\mathcal{C}$, llamada \textbf{reparametrización} de $\alpha$ a través del cambio de parámetros $h$. \\
	(2) Si $\alpha$ es inyectiva entonces $\beta$ también es inyectiva. \\
	(3) Si $\alpha$ es una inmersión entonces $\beta$ también es una inmersión.
\end{propobox}

\begin{propobox}[label=propo:existencia_cambio_param]
	Si $\alpha : I \rightarrow \reals^3$ y $\beta : J \rightarrow \reals^3$ son dos inmersiones inyectivas de una misma curva $\mathcal{C}\subseteq\reals^3$ entonces existe un cambio de parámetros $h: I \rightarrow \reals$ con $h(I)=J$ tal que $\beta = \alpha \circ h^{-1}$
\end{propobox}

\begin{defbox}{Integral de un campo escalar sobre una curva}
	Sea $\mathcal{C}\subset\reals^3$ una curva regular simple y $f: U \subset \reals^3 \rightarrow \reals$ un campo escalar continuo en un abierto $U$ que contiene a $\mathcal{C}$. Si $\alpha : [a,b] \rightarrow \reals^3$ es una inmersión de la curva $\mathcal{C}$ definimos la integral del campo escalar $f$ sobre la curva $\mathcal{C}$ como:
	\begin{equation}
		\boxed{\int_{\mathcal{C}}{fds}\defeq \int_a^b{f(\alpha (t)) ~||\alpha '(t) ||dt}}
		\label{eq:int_c_f}
	\end{equation}
\end{defbox}

\begin{teobox}{Independencia respecto a la parametrización}
	Sea $\mathcal{C} \subset \reals^3$ una curva regular simple y $f: U \subset \reals^3 \rightarrow \reals$ un campo escalar continuo en un abierto $U$ que contiene a $\mathcal{C}$. \\
	Si $\alpha :[a,b] \rightarrow \reals^3$ y $\beta : [c,d] \rightarrow \reals^3$ son dos inmersiones inyectivas de la curva $\mathcal{C}$ entonces se cumple que: \\
	\[
		\int_a^b{f(\alpha (t))||\alpha '(t)||dt} = \int_c^d{f(\beta (s))||\beta '(s)||ds}
	\]
	\tcblower
\end{teobox}
\underline{Demostración} \\
Por la proposición \ref{propo:reparametrizacion} y \ref{propo:existencia_cambio_param} sabemos que $\beta$ es una reparametrización de $\alpha$ a través de un cambio de parámetros $h: [a,b] \rightarrow [c,d]$ esto es: \\
\[
	\alpha = \beta \circ h
\]
Primer caso: Si $h'(t) > 0 ~~~ \forall t \in [a,b]$ la función $h: [a,b] \rightarrow [c,d]$ es creciente y se cumple que
\[
	\left\{ \begin{array}{l}
		h(a) = c \\
		h(b) = d
	\end{array} \right.
\]
Luego \\
\[
	\begin{array}{ll}
		\int_a^b{f(\alpha (t)) ||\alpha '(t)||dt} & = \int_a^b{f(\beta (h(t))) ||[\beta (h(t))]'||dt}    \\
		                                          & = \int_a^b{f(\beta (h(t))) ||\beta '(h(t))h'(t)||dt} \\
		                                          & = \int_a^b{f(\beta (h(t))) ||\beta '(h(t))||h'(t)dt} \\
		                                          & = \int_{h(a)}^{h(b)}{f(\beta (s) ||\beta '(s)||ds}   \\
		                                          & = \int_{c}^{d}{f(\beta (s) ||\beta '(s)||ds}         \\
	\end{array}
\]
Segundo caso: Si $h'(t) < 0 ~~~ \forall t \in [a,b]$ la función $h: [a,b] \rightarrow [c,d]$ es decreciente y se cumple que
\[
	\left\{ \begin{array}{l}
		h(a) = d \\
		h(b) = c
	\end{array} \right.
\]
Luego \\
\[
	\begin{array}{ll}
		\int_a^b{f(\alpha (t)) ||\alpha '(t)||dt} & = \int_a^b{f(\beta (h(t))) ||[\beta (h(t))]'||dt}     \\
		                                          & = \int_a^b{f(\beta (h(t))) ||\beta '(h(t))h'(t)||dt}  \\
		                                          & = -\int_a^b{f(\beta (h(t))) ||\beta '(h(t))||h'(t)dt} \\
		                                          & = -\int_{h(a)}^{h(b)}{f(\beta (s) ||\beta '(s)||ds}   \\
		                                          & = -\int_{d}^{c}{f(\beta (s) ||\beta '(s)||ds}         \\
		                                          & = \int_{c}^{d}{f(\beta (s) ||\beta '(s)||ds}          \\
	\end{array}
\]

\section{Integrales de campos vectoriales sobre curvas}

\begin{defbox}{Espacio de direcciones tangente}
	Sea $\mathcal{C}$ una curva regular y simple en $\reals^3$ y $p\in \mathcal{C}$. \\
	Si $\alpha : I \rightarrow \reals^3$ es una inmersión inyectiva de la curva $\mathcal{C}$ existe un único $t\in I$ tal que
	\[
		p = \alpha (t)
	\]
	Definimos el \textbf{espacio de direcciones tangentes} $\mathcal{C}$ en $p$, y lo indicamos por $T_p(\mathcal{C})$, al subespacio vectorial de $\reals^3$ generado por $\alpha (t)$, esto es
	\[
		T_p(\mathcal{C}) \defeq [\alpha '(t)]
	\]
	Y diremos que \\
	$v \ne 0$ es una \textbf{dirección tangente} a $\mathcal{C}$ en $p \iff v \in T_p(\mathcal{C})$
\end{defbox}

\begin{defbox}{Orientación}
	Sea $\mathcal{C}$ una curva regular y simple en $\reals^3$. \\
	Un campo vectorial $\vec{T}: \mathcal{C} \rightarrow \reals^3$ es una \textbf{orientación} en $\mathcal{C} \iff$ cumple las siguientes propiedades:
	\begin{enumerate}
		\item $\vec{T}$ es continuo \\
		\item $\vec{T}$ es un versor (esto es $||\vec{T}||=1$) \\
		\item $\vec{T}(p)$ es una dirección tangente a $\mathcal{C}$ en $p$ (esto es $\vec{T}(p)\in T_p(\mathcal{C})$)
	\end{enumerate}
	Una \textbf{curva orientada} es una curva en la cual hemos elegido una orientación.
\end{defbox}

\begin{defbox}{Integral de un campo vectorial sobre una curva}
	Sea $\mathcal{C} \subset \reals^3$ una curva regular simple orientada por el campo continuo $\vec{T}$ de versores tangentes. \\
	Consideremos un campo vectorial $\vec{X}: U \subseteq \reals^3 \rightarrow \reals^3$ continuo en un abierto $U$ que contiene a $\mathcal{C}$. Definimos la integral del campo vectorial $\vec{X}$ a lo largo de la curva $\mathcal{C}$ como:
	\[
		\int_{\mathcal{C}}{\vec{X}ds} \defeq \int_{\mathcal{C}}{\vec{X}\cdot\vec{T}ds}
	\]
\end{defbox}

% Formula de calculo
\begin{propobox}
	Sea $\mathcal{C} \subset \reals^3$ una curva regular simple orientada por el campo continuo $\vec{T}$ de versores tangentes y $\vec{X}:U\subseteq\reals^3 \rightarrow\reals^3$ un campo vectorial continuo en un abierto $U$ que contiene a $\mathcal{C}$. \\
	Si $\alpha :[a,b] \rightarrow \reals^3$ es una inmersión de la curva $\mathcal{C}$ entonces
	\begin{equation}
		\boxed{
		\int_{\mathcal{C}}{\vec{X}ds} = \pm \int_a^b{\vec{X}(\alpha (t))\cdot \alpha '(t)dt}
		}
		\label{eq:int_c_vec}
	\end{equation}
\end{propobox}
\underline{Demostración} \\
\[
	\begin{array}{ll}
		\int_{\mathcal{C}}{\vec{X}ds} & \defeq \int_{\mathcal{C}}{\vec{X}\cdot\vec{T}ds}                                                \\
		                              & = \int_a^b{(\vec{X}(\alpha (t))\cdot \vec{T}(\alpha (t)))||\alpha '(t))||dt}                    \\
		                              & =\pm \int_a^b{(\vec{X}(\alpha (t))\cdot \frac{\alpha '(t)}{||\alpha '(t)||})||\alpha '(t))||dt} \\
		                              & = \pm \int_a^b{\vec{X}(\alpha (t))\cdot \alpha '(t)dt}
	\end{array}
\]

\chapter{Integrales sobre superficies}
\section{Integrales de campos escalares sobre superficies}
\begin{defbox}{Parametrización}
	Llamaremos \textbf{parametriación} a toad función $\varphi : D \subseteq \reals^2 \rightarrow \reals^3$, donde $D$ es un conjunto abierto y conexo en $\reals^2$.
\end{defbox}

\begin{defbox}{Superficie suave}
	Sea $\mathcal{S}$ un conjunto no vacío de $\reals^3$. \\
	Diremos que \\
	$\mathcal{S}$ es una \textbf{superficie suave} $\iff$ \underline{existe} una parametrización \\ $\varphi : D \subseteq \reals^2 \rightarrow \reals^3$ de clase $C^1$ tal que $\mathcal{S}=\varphi (D)$ siendo $D$ un conjunto conexo de $\reals^2$.
\end{defbox}

\begin{defbox}{Inmersión}
	Consideremos una parametrización $\varphi : D \subseteq \reals^2 \rightarrow \reals^3$. \\
	Diremos que\\
	$\varphi$ es una \textbf{inmersión}
	$\iff \left\{\begin{array}{l}
			1. ~\varphi \text{ es de clase } C^1 \\
			2. ~\varphi_u(u,v)\times \varphi_v(u,v) \ne 0
		\end{array}\right.$
\end{defbox}

\begin{defbox}{Superficie regular}
	Sea $\mathcal{S}$ un conjunto no vacío de $\reals^3$. \\
	Diremos que \\
	$\mathcal{S}$ es una \textbf{superficie regular} $\iff$ \underline{existe} una inmersión \\ $\varphi : D \subseteq \reals^2 \rightarrow \reals^3$ tal que $\mathcal{S} = \varphi (D)$
\end{defbox}

\begin{defbox}{Superficie simple}
	Sea $\mathcal{S}$ un conjunto no vacío de $\reals^3$. \\
	Diremos que \\
	$\mathcal{S}$ es una \textbf{superficie simple} $\iff$ \underline{existe} una parametrización \\ $\varphi : D \subseteq \reals^2 \rightarrow \reals^3$ de clase $C^1$ tal que $\mathcal{S}=\varphi (D)$ con $\varphi$ \textbf{inyectiva en $D$}.
\end{defbox}

\begin{defbox}{Integral de un campo escalar sobre una superficie}
	Sea $\mathcal{S} \subset \reals^3$ una superficie regular y simple y $f : U \subset \reals^3 \rightarrow \reals$ un campo escalar continuo en un abierto $U$ que contiene a $\mathcal{S}$. Si $\varphi : D \subseteq \reals^2 \rightarrow \reals^3$ es una inmersión inyectiva que cuber a $\mathcal{S}$ definimos la integral del campo escalar $f$ sobre la superficie $\mathcal{S}$ como:
	\begin{equation}
		\boxed{
			\iint_{\mathcal{S}}{fdS} \defeq \iint_{D}{f(\varphi (u, v))||\varphi_u (u,v) \times \varphi_v (u,v)||dudv}
		}
		\label{eq:int_s_f}
	\end{equation}
\end{defbox}

% Lema de algebra v_1 x v_2 = det(a) (w_1 x w_2)
\begin{propobox}[label=propo:lema_alg_lin]
	Sean $v_1, v_2, w_1, w_2$ vectores en $\reals^3$ y $A$ una matriz real $2\times 2$ y las matrices $V$ y $W$ siendo conformadas por los vectores $v_1, v_2$ y $w_1, w_2$ puestos como columnas respectivamente, tales que \\
	\[
		V = WA
	\]
	entonces
	\[
		v_1 \times v_2 = det(A) (w_1 \times w_2)
	\]
\end{propobox}

% unas cosas necias con vectoriales de las derivadas del phi
\begin{propobox}
	Si $\varphi$ es una reparametrización de $\psi$ a través del cambio de parámetros $h$ entonces
	\[
		\varphi_u (u,v) \times \varphi_v (u,v) = det(\mathds{J}h(u,v)) (\psi_s (s,t) \times \psi_t (s,t))
	\]
	donde $(s,t) = h(u,v)$.
\end{propobox}
\underline{Demostración} \\
Tenemos que \\
\[
	\varphi = \psi \circ h
\]
y por la regla de la cadena
\[
	\mathds{J}\varphi (u,v) = \mathds{J}\psi (h (u,v) ) \cdot \mathds{J}h(u,v)
\]
\[
	\iff \mathds{J}\varphi (u,v) = \mathds{J}\psi (s,t) \cdot \mathds{J}h(u,v)
\]
Aplicando la proposición \ref{propo:lema_alg_lin} conluimos \\
\[
	\varphi_u (u,v) \times \varphi_v (u,v) = det(\mathds{J}h(u,v)) (\psi_s (s,t) \times \psi_t (s,t))
\]

\begin{teobox}{Independencia respecto a la parametrización}
	Sea $\mathcal{S} \subset \reals^3$ una superficie regular simple y $f: U \subseteq \reals^3 \rightarrow \reals$ un campo escalar continuo en un abierto $U$ que contiene a $\mathcal{S}$. \\
	Si $\varphi : D \subset \reals^2 \rightarrow \reals^3$ y $\psi : E \subset \reals^2 \rightarrow \reals^3$ son dos inmersiones inyectivas de la superficie $\mathcal{S}$ entonces se cumple \\
	\[
		\iint_D{f(\varphi (u,v) ) ||\varphi_u (u,v) \times \varphi_v (u,v)||~dudv}
	\]
	\[ = \]
	\[
		\iint_E{f(\psi (s,t) ) ||\psi_s (s,t) \times \psi_t (s,t)||~dsdt}
	\]
\end{teobox}
\underline{Demostración} \\
Tenemos que $\psi$ es una reparametrización de $\varphi$ a través de un cambio de parámetros $h: D \rightarrow E$, esto es:
\[
	\psi = \varphi \circ h
\]
Aplicando el cambio de variables (en integral doble)
\[
	\left\{\begin{array}{l}
		(s,t) = h(u,v) \\
		dsdt = |det(\mathds{J}h(u,v))|dudv
	\end{array}\right.
\]
se tiene que \\
\[
	\iint_E{f(\psi (s,t) ) ||\psi_s (s,t) \times \psi_t (s,t)|| dsdt} =
	\iint_E{f(\psi (h(u,v)) ||\psi_s h((u,v)) \times \psi_t (h(u,v))|||det(\mathds{J}h(u,v))| dudv}
\]
pero como $\varphi = \psi \circ h$
\[
	f(\psi (h (u,v) ) ) = f(\varphi (u,v) )
\]
y utilizando la proposición \ref{propo:lema_alg_lin} se deduce que
\[
	||\psi_s (h(u,v)) \times \psi_t (h(u,v))|| = ||\varphi_u (u,v) \times \varphi_v (u,v)||
\]
sustituyendo obtenemos
\[
	\iint_D{f(\varphi (u,v) ) ||\varphi_u (u,v) \times \varphi_v (u,v)||~dudv} =
	\iint_E{f(\psi (s,t) ) ||\psi_s (s,t) \times \psi_t (s,t)||~dsdt}
\]
\section{Integrales de campos vectoriales sobre superficies}

\begin{defbox}{Espacio de direcciones normales}
	Sea $\mathcal{S}$ una superficie regular simple y orientable en $\reals^3$ y $p \in \mathcal{S}$.\\
	Si $\varphi : D \rightarrow \reals^3$ es una inmersión inyectiva de la superficie $\mathcal{S}$ existe un único $(u,v)\in D$ tal que
	\[
		p=\varphi (u,v)
	\]
	Definimos el \textbf{espacio de direcciones normales} $\mathcal{S}$ en $p$, y llo indicamos por $[T_p(\mathcal{S})]^\bot$, al subespacio vectorial de $\reals^3$ generado por $\varphi_u (u,v) \times \varphi_v (u,v)$, eso es
	\[
		[T_p(\mathcal{S})]^\bot \defeq gen(\varphi_u (u,v) \times \varphi_v (u,v))
	\]
	Y diremos que \\
	$\vec{n} \ne 0$ es una \textbf{dirección normal} a $\mathcal{S}$ en $p \iff \vec{n} \in [T_p(\mathcal{S})]^\bot$
\end{defbox}

\begin{defbox}{Orientación}
	Sea $\mathcal{S}$ una superficie regular simple y orientable en $\reals^3$ y $p \in \mathcal{S}$.\\
	Un campo vectorial $\vec{n} : \mathcal{C} \rightarrow \reals^3$ es una \textbf{orientación} en $\mathcal{S} \iff$ cumple las siguientes propiedades:
	\begin{enumerate}
		\item $\vec{n}$ es continuo
		\item $\vec{n}$ es un versor (esto es $||\vec{n}|| = 1$
		\item $\vec{n} (p)$ es un vector normal a $\mathcal{C}$ en $p$ (esto es $\vec{n}(p)\in [T_p(\mathcal{S})]^\bot$
	\end{enumerate}
\end{defbox}

\begin{defbox}{Integral de un campo vectorial sobre una superficie}
	Sea $\mathcal{S} \subset \reals^3$ una superficie regular simple orientada por el campo continuo $\vec{n}$ de versores normales.
	consideremos un campo vectorial $\vec{X} : U \subseteq \reals^3 \rightarrow \reals^3$ continuo en un abierto $U$ que contiene a $\mathcal{S}$.
	Definimos la integral del campo vectorial $\vec{X}$ sobre la superficie $\mathcal{S}$ como:
	\begin{equation}
		\boxed{
			\iint_{\mathcal{S}}{\vec{X}dS} \defeq \iint_{\mathcal{S}}{\vec{X}\cdot \vec{n}dS}
		}
		\label{eq:int_s_vec}
	\end{equation}
\end{defbox}

% Formula de calculo usando parametrizaciones
\begin{propobox}
	Sea $\mathcal{S} \subset \reals^3$ una superficie regular orientada por el campo continuo $\vec{n}$ de versores normales a $\mathcal{S}$ y $\vec{X} : U \subseteq \reals^3 \rightarrow \reals^3$ un campo vectorial continuo en un abierto $U$ que contiene a $\mathcal{S}$.
	Si $\varphi : D \subset \reals^2 \rightarrow \reals^3$ es una inmersión inyectiva de la $\mathcal{S}$ entonces
	\begin{equation}
		\boxed{
		\iint_{\mathcal{S}}{\vec{X}dS} =
		\pm \iint_D{\vec{X}(\varphi (u,v)) \cdot (\varphi_u (u,v) \times \varphi_v (u,v)) dudv}
		}
		\label{eq:form_int_s_vec}
	\end{equation}
	donde el signo $\pm$ depende de si la superfice es compatible o no con la orientación de $\mathcal{S}$.
\end{propobox}
\underline{Demostración} \\
\[
	\begin{array}{ll}
		\iint_\mathcal{S}{\vec{X}dS} & \defeq \iint_{\mathcal{S}}{\vec{X}\cdot\vec{n}dS}                                                                                                                                     \\
		                             & ~=~ \iint_D{\vec{X}(\varphi (u,v))\cdot \vec{n}(\varphi (u,v)) ~||\varphi_u (u,v) \times \varphi_v (u,v)||dudv}                                                                       \\
		                             & ~= \pm \iint_D{\vec{X}(\varphi (u,v))\cdot \frac{\varphi_u (u,v) \times \varphi_v (u,v)}{||\varphi_u (u,v) \times \varphi_v (u,v)||} ~||\varphi_u (u,v) \times \varphi_v (u,v)||dudv} \\
		                             & ~= \pm \iint_D{\vec{X}(\varphi (u,v)) \cdot (\varphi_u (u,v) \times \varphi_v (u,v)) dudv}
	\end{array}
\]

\chapter{Teoremas clásicos del cálulo vectorial}

% semejante apampanado el que se puso a demostrar esto
\begin{teobox}{de la curva de Jordan}
	Toda curva cerrada y simple $\mathcal{C}$ en $\reals^2$ divide al propio $\reals^2$ en dos conjuntos abiertos conexos y disjuntos $A_1$ y $A_2$ cuya frontera común es la curva $\mathcal{C}$. \\
	Además uno de ellos es acotada (se denomina ``interior`` de $\mathcal{C}$) y el otroes no acotada (se denomina ``exterior`` de $\mathcal{C}$).
\end{teobox}


\begin{teobox}{Green}
	Sea $\mathcal{C}$ una curva cerrada simple orientada en \underline{\textbf{sentido antihorario}} y sea $D$ la unión de la curva $\mathcal{C}$ con la región conexa ``interior`` a $\mathcal{C}$. \\
	Si el campo $\vec{X} = (P,Q)$ es de clase $C^1$ en un abierto que contiene a $D$ se cumple que
	\begin{equation}
		\boxed{
			\int_{\mathcal{C}^+}{\vec{X}ds} = \iint_D{(Q_x - P_y)dxdy}
		}
		\label{eq:teo_green}
	\end{equation}
\end{teobox}
\begin{proof} para regiones elementales \\
Si consideramos los campos $\vec{P} \defeq (P,0)$ y $\vec{Q} \defeq (0,Q)$ se tiene que
\[
	\vec{X} = (P,Q) = (P,0) + (0,Q) = \vec{P} + \vec{Q}
\]
y por la propiedad de linealidad de las integrales sobre curvas se cumple que
\[
	\int_{\mathcal{C}^+}{\vec{X}ds} = \int_{\mathcal{C}^+}{\vec{P}ds} + \int_{\mathcal{C}^+}{\vec{Q}ds}
\]
Por otro lado, por la propiedad de linealidad de las integrales dobles, también se tiene que
\[
	\iint_D{(Q_x - P_y)dxdy} = \iint_D{Q_xdxdy} - \iint_D{P_ydxdy}
\]
Por lo tanto bastará con probar que
\[
	\int_{\mathcal{C}^+}{\vec{P}ds} = - \iint_D{P_ydxdy}
\]
\[
	\int_{\mathcal{C}^+}{\vec{Q}ds} = \iint_D{Q_xdxdy}
\]

\underline{Primera parte} \\
Como $D$ es elemental lo podemos exresar de la siguiente manera:
\[
	D= \left\{ (x,y)\in \reals^2 : a \le x \le b, f_1(x) \le y \le f_2(x)\right\}
\]
donde $f_1$ y $f_2$ son funciones de clase $C^1$ en el intervalo $[a,b]$.
Evaluamos la integral doble en $D$
\[
	-\iint_D{P_y(x,y)dxdy} =
	-\int_a^b{\left[\int_{f_1(x)}^{f_2(x)}{P_y(x,y)dy} \right]dx} =
	-\int_a^b{\left[ \left.{P(x,y)}\right|_{f_1(x)}^{f_2(x)} \right]dx} =
\]
\[
	= -\int_a^b{\left[ P(x,f_2(x))-P(x,f_1(x))\right]dx}
	= \boxed{
	\int_a^b{\left[ P(x,f_1(x))-P(x,f_2(x))\right]dx}
	}
\]
Por otro lado la curva $\mathcal{C}$ se puede descomponer en dos curvas $\mathcal{C} = \mathcal{C}_2 \cup \mathcal{C}_2$ y se cumple que
\[
	\int_{\mathcal{C}}{\vec{P}ds} =
	\int_{\mathcal{C}_1}{\vec{P}ds} +
	\int_{\mathcal{C}_2}{\vec{P}ds}
\]
Siendo la curva $\mathcal{C}_1$ el gráfico de la función $f_1$ la podemos parametrizar por
\[
	\alpha_1 : \alpha_1(x) = (x,f_1(x))
\]
con $x\in [a,b]$ que es compatible con la orientación de $\mathcal{C}_1$. \\
De esta manera
\[
	\int_{\mathcal{C}_1}{\vec{P}ds} =
	\int_a^b{\vec{P}(\alpha_1(x))\cdot \alpha_1'(x)dx} =
	\int_a^b{(P(\alpha_1(x),0)\cdot (1,f_1'(x))dx} =
	\int_a^b{(P(x,f_1(x)))dx}
\]
De forma análoga,
siendo la curva $\mathcal{C}_2$ el gráfico de la función $f_2$ la podemos parametrizar por
\[
	\alpha_2 : \alpha_2(x) = (x,f_2(x))
\]
con $x\in [a,b]$ que \underline{no} es compatible con la orientación de $\mathcal{C}_2$. \\
Así
\[
	\int_{\mathcal{C}_2}{\vec{P}ds} =
	-\int_a^b{(P(x,f_2(x)))dx}
\]
Entonces
\[
	\int_{\mathcal{C}}{\vec{P}ds} =
	\int_{\mathcal{C}_1}{\vec{P}ds} +
	\int_{\mathcal{C}_2}{\vec{P}ds} =
	\int_a^b{(P(x,f_1(x)))dx}
	-\int_a^b{(P(x,f_2(x)))dx}
\]
\[
	=
	\boxed{
	\int_a^b{\left[P(x,f_1(x))-P(x,f_2(x))\right]dx}
	}
\]
Podemos concluir que
\[
	\int_{\mathcal{C}^+}{\vec{P}ds} = - \iint_D{P_ydxdy}
\]
El segundo caso es análogo.
\end{proof}


% mi gran y querido
\begin{defbox}{Operador de Hamilton}
	El \textbf{operador de Hamilton $\nabla$ ``nabla``} \\
	\[
		\nabla \defeq \left(\frac{\partial}{\partial{x}},\frac{\partial}{\partial{y}},\frac{\partial}{\partial{z}}\right)
	\]
	es un operador que resulta util para expresar las nociones que veremos más adelante.
\end{defbox}

\begin{defbox}{Rotor}
	Sea $\vec{X} : U \subseteq \reals^3 \rightarrow \reals^3$ un campo vectorial de calse $C^1$ en un abierto $U$ de $\reals^3$. \\
	Se define el \textbf{rotor} del campo $\vec{X} = (P,Q,R)$ como el campo vectorial $Rot(\vec{X}) : U \subseteq \reals^3 \rightarrow \reals^3$ tal que
	\begin{equation}
		Rot(\vec{X}) \defeq \left(
		\frac{\partial{R}}{\partial{y}}-\frac{\partial{Q}}{\partial{z}},
		\frac{\partial{P}}{\partial{z}}-\frac{\partial{R}}{\partial{x}},
		\frac{\partial{Q}}{\partial{x}}-\frac{\partial{P}}{\partial{y}}
		\right)
		\label{eq:rotor}
	\end{equation}
	\underline{Observación} \\
	\[
		Rot(\vec{X}) = \nabla \times \vec{X}
	\]
\end{defbox}

\begin{teobox}{Stokes}
	Sea $\mathcal{S}$ una superficie con borde $\partial{\mathcal{S}}$ la cual está orientada con el campo de versores normales $\vec{n}$. Si un campo $\vec{X}= (P,Q,R)$ es de clase $C^1$ en un abierto que contiene a la superficie $\mathcal{S}$ entonces se cumple que
	\begin{equation}
		\iint_{\mathcal{S}}{Rot(\vec{X})dS} =
		\int_{\partial{\mathcal{S}}}{\vec{X}ds}
		\label{eq:teo_stokes}
	\end{equation}
	donde la curva borde $\partial{\mathcal{S}}$ está orientada con la orientación inducida por $\vec{n}$.
\end{teobox}
\begin{proof}
	Vamos a realizar la demostración para el caso en el que la superficie es el grafico de una función. \\
	Supongamos que:
	\[
		\mathcal{S} = \left\{ (x,y,z)\in \reals^3 : z = f(x,y), (x,y) \in D\right\}
	\]
	donde $f: U \subseteq \reals^2 \rightarrow \reals$ es de clase $C^2$ y $D$ es un abierto contenido en $U$ tal que su frontera es una curva $\mathcal{C}$ también contenida en $U$.

	Sin pérdida de generalidad orientamos la superficie con la normal $\vec{n}$ exterior (tercera componente positiva). \\
	$\vec{n}$ induce una orientación en $\partial \mathcal{S}$ y se orienta a $\mathcal{C}$ de la misma manera. \\
	($\mathcal{C}$ es la proyección sobre el plano $xy$ de la curva $\partial\mathcal{S}$ por construcción)

	\underline{Parametrización de la superficie} \\
	\[
		\varphi (x,y) = (x, y, f(x,y))
	\]
	con $(x,y) \in D$ que induce la normal en la superficie:
	\[
		\varphi_x \times \varphi_y = (1, 0, f_x)\times (0, 1, f_y) = (-f_x, -f_y, 1)
	\]
	con tercer componente positiva $\Rightarrow \varphi$ es compatible con la normal $\vec{n}$.

	\underline{Parametrización de las curvas} \\
	\[
		\alpha (t) = (x(t), y(t))
	\]
	con $t \in [a,b]$ es una parametrización genérica de la curva $\mathcal{C}$. \\
	Por lo tanto, $\gamma = \varphi \circ \alpha$ es una parametrización de $\partial\mathcal{S}$
	\[
		\gamma (t) = \varphi (\alpha (t)) =
		\varphi (x(t),~y(t)) =
		(x(t),~y(t),~f(x(t),y(t)))
	\]
	Luego \\
	\begin{gather*}
		\begin{aligned}
			\int\limits_{\partial\mathcal{S}}{\vec{X}ds}
			 & = \int\limits_a^b{\vec{X}(\gamma (t)) \cdot \gamma '(t)dt}                                                                                                                                   \\
			 & \left(
			\begin{aligned}
					\vec{X}(\gamma (t)) & = (P(\gamma (t)),~Q(\gamma (t)),~R(\gamma (t)))                               & \text{pues }\vec{X} = (P,Q,R) \\
					                    & = (P(\varphi (\alpha (t))),~Q(\varphi (\alpha (t))),~R(\varphi (\alpha (t))))
					                    & \text{pues } \gamma = \varphi \circ \alpha                                                                    \\
					\gamma '(t)         & = \left(x'(t),~y'(t),~f_x(x(t), y(t))x'(t) + f_y(x(t), y(t))y'(t) \right)                                     \\
					                    & = \left(x'(t),~y'(t),~f_x(\alpha (t))x'(t) + f_y(\alpha (t))y'(t) \right)
					                    & \text{pues } \alpha (t) = (x(t),~y(t))
				\end{aligned}
			\right)                                                                                                                                                                                         \\
			 & = \int\limits_a^b{(~P(\varphi (\alpha (t))),~Q(\varphi (\alpha (t))),~R(\varphi (\alpha (t)))~) \cdot \left(x'(t),~y'(t),~f_x(\alpha (t))x'(t) + f_y(\alpha (t))y'(t) \right)dt}             \\
			 & \text{operamos y agrupamos en } x'(t) \text{ e } y'(t)                                                                                                                                       \\
			 & = \int\limits_a^b{\left[ P(\varphi (\alpha (t))) + R(\varphi (\alpha (t)))f_x(\alpha(t))\right]x'(t) + \left[ Q(\varphi (\alpha (t))) + R(\varphi (\alpha (t)))f_y(\alpha(t))\right]y'(t)dt} \\
			 & = \int\limits_a^b{\left( P(\varphi (\alpha (t))) + R(\varphi (\alpha (t)))f_x(\alpha(t)),~Q(\varphi (\alpha (t))) + R(\varphi (\alpha (t)))f_y(\alpha(t))\right)\cdot (x'(t),y'(t))dt}       \\
			% abuse pepe abuse
			 & = \int\limits_a^b{(P\circ \varphi + (R \circ \varphi)f_x,~Q\circ \varphi + (R \circ \varphi)f_y)(\alpha (t))\cdot \alpha '(t) dt}                                                            \\
			 & \text{vamos a simplificar la notación usando } \vec{Y} = (F,~G)\defeq (P\circ \varphi + (R \circ \varphi)f_x,~Q\circ \varphi + (R \circ \varphi)f_y)                                         \\
			 & = \int\limits_a^b{(F,G)(\alpha (t))\cdot \alpha '(t)dt}                                                                                                                                             \\
			 & = \int\limits_a^b{\vec{Y}(\alpha (t))\cdot \alpha '(t)dt}                                                                                                                                           \\
			 & = \int\limits_{\mathcal{C}}{\vec{Y}ds}                                                                                                                                           \\
			 & \text{aplicando el Teorema de Green} \\
			 & = \iint\limits_D{G_x-F_ydxdy}
		\end{aligned}
	\end{gather*}
	Calculemos las derivadas $G_x$ y $F_y$
	\[
		G = (Q \circ \varphi) + (R \circ \varphi)f_y
		\Rightarrow G_x = (Q \circ \varphi)_x + (R \circ \varphi)_xf_y+ (R \circ \varphi)f_{xy}
	\]
	Pero 
	\[
		(Q\circ \varphi) = (Q(x,y,f(x,y))) 
		\Rightarrow 
		(Q\circ \varphi)_x = 
		(Q_x\circ\varphi) + (Q_z\circ\varphi)f_x
	\]
	y 
	\[
		(R\circ\varphi) = (R(x,y,f(x,y))) 
		\Rightarrow 
		(R\circ\varphi)_x = 
		(R_x\circ\varphi) + (R_z\circ\varphi)f_x
	\]
	entonces
	\[
		G_x = (Q_x\circ\varphi) + (Q_z\circ\varphi)f_x + 
		((R_x\circ\varphi) + (R_z\circ\varphi)f_x)f_y+ 
		(R \circ \varphi)f_{xy}
	\]
	De la misma forma se tiene que
	\[
		F_y = (P_y \circ \varphi) + (P_z \circ \varphi)f_y +
		((R_y\circ\varphi)+(R_z\circ\varphi)f_y)f_x+
		(R\circ\varphi)f_{xy}
	\]
	de donde:
	\begin{gather*}
		\begin{aligned}
			G_x - F_y & = [(Q_z\circ\varphi)-(R_y\circ\varphi)]f_x + [(R_x\circ\varphi)-(P_z\circ\varphi)]f_y+[(Q_x\circ\varphi)-(P_y\circ\varphi)] \\ 
					  & = [(Q_z - R_y)\circ\varphi]f_x + [(R_x-P_z)\circ\varphi]f_y+(Q_x-P_y)\circ\varphi \\ 
					  & = [(R_y - Q_z)\circ\varphi](-f_x) + [(R_x-P_z)\circ\varphi](-f_y)+(Q_x-P_y)\circ\varphi (1) \\
					  & = ([(R_y - Q_z)\circ\varphi] + [(R_x-P_z)\circ\varphi] + (Q_x-P_y)\circ\varphi)\cdot (-f_x,-f<y,1) \\
					  & = ([(R_y - Q_z)\circ\varphi] + [(R_x-P_z)\circ\varphi] + (Q_x-P_y)\circ\varphi)\cdot (\varphi_x \times \varphi_y) \\
					  & (\text{observar que siendo }\vec{X} = (P,Q,R) \text{ entonces }Rot(\vec{X})=(R_y-Q_z,P_z-R_x,Q_x-P_y)) \\
					  & = Rot(\vec{X}\circ\varphi )\cdot (\varphi_x \times\varphi_y)
		\end{aligned}
	\end{gather*}
	De esta manera
	\[
		\iint\limits_D{G_x - F_y dxdy} = \iint\limits_D{Rot(\vec{X}\circ\varphi)\cdot (\varphi_x\times\varphi_y)dxdy} = \iint\limits_\mathcal{S}{\vec{X}dS}
	\]
\end{proof}

\begin{defbox}{Divergencia}
	Sea $\vec{X}: U \subseteq \reals^3 \rightarrow \reals^3$ un campo vectorial de clase $C^1$ en un abierto $U$ de $\reals^3$. \\
	Se define la \textbf{divergencia} del campo $\vec{X} = (P,Q,R)$ como el campo escalar $Div(\vec{X}): U \subseteq \reals^3 \rightarrow \reals$ tal que
	\begin{equation}
		Div(\vec{X}) \defeq \frac{\partial{P}}{\partial{x}}+\frac{\partial{Q}}{\partial{y}}+\frac{\partial{R}}{\partial{z}}
		\label{eq:divergencia}
	\end{equation}
	\underline{Observación}
	\[
		Div(\vec{X}) = \nabla \cdot \vec{X}
	\]
\end{defbox}

\begin{teobox}{Gauss}
	Sea $\mathcal{S}$ una superficie cerrada de $\reals^3$ que es la frontera de un sólido $E$. \\
	Si el campo $\vec{X} = (P,Q,R)$ es de clase $C^1$ en un abierto que contiene a $E \cup \mathcal{S}$ se cumple que
	\begin{equation}
		\iint_{\mathcal{S}}{\vec{X}dS} =
		\iiint_{\Omega}{Div(\vec{X})dxdydz}
		\label{eq:teo_gauss}
	\end{equation}
	donde la superficie $\mathcal{S}$ debe estar orientada con la \textbf{normal exterior} al sólido $E$.
\end{teobox}
\begin{proof} para sólidos elementales\\
Si consideramos los campos $\vec{P}\defeq (P,0,0)$,$\vec{Q}\defeq (0,Q,0)$,$\vec{R}\defeq (0,0,R)$ se tiene que
\[
	\vec{X} = (P,Q,R) = \vec{P} + \vec{Q} + \vec{R}
\]
Por la propiedad de linealidad de las integrales de superficie se cumple que:
\[
	\iint\limits_{\mathcal{S}^+}{\vec{X}dS} =
	\iint\limits_{\mathcal{S}^+}{\vec{P}dS} +
	\iint\limits_{\mathcal{S}^+}{\vec{Q}dS} +
	\iint\limits_{\mathcal{S}^+}{\vec{R}dS}
\]
Por otro lado, por la linealidad de las integrales triples se cumple:
\[
	\iiint\limits_E{Div(\vec{X})dxdydz} =
	\iiint\limits_E{P_x + Q_y + R_z dxdydz} = 
\]
\[
	\iiint\limits_E{P_x dxdydz} + 
	\iiint\limits_E{Q_y dxdydz} + 
	\iiint\limits_E{R_z dxdydz}
\]
Bastará con probar que:
\[
	\iint\limits_{\mathcal{S}^+}{\vec{P}dS} =
	\iiint\limits_E{P_x dxdydz}
\]
\[
	\iint\limits_{\mathcal{S}^+}{\vec{Q}dS} =
	\iiint\limits_E{Q_y dxdydz}
\]
\[
	\iint\limits_{\mathcal{S}^+}{\vec{R}dS} =
	\iiint\limits_E{R_z dxdydz}
\]
Demostraremos la tercer igualdad. \\ 
Siendo $E$ un sólido elemental lo podemos representar de la siguiente manera: 
\[
	E = \{ (x,y,z)\in \reals^3: (x,y)\in D, f_1(x,y)< z < g_1(x,y)\}
\]
donde $f_1$ y $g_1$ son funciones de clase $C^1$ en el abierto $D_1$ \\ 
Evaluando la integral triple en $E$:
\begin{gather*}
	\begin{aligned}
		\iiint\limits_E{R_z(x,y,z)dxdydz} & = \iint\limits_D{\left( \int\limits_{f_1(x,y)}^{g_1(x,y)}{R_x(x,y,z)dz}\right)dxdy} \\ 
										  & = \iint\limits_D{R(x,y,g_1(x,y))-R(x,y,f_1(x,y))dxdy}
	\end{aligned}
\end{gather*}
Por otro lado la superficie $\mathcal{S}^+$ se puede descomponer en $\mathcal{S}_1$ (gráfico de $f_1$) y $\mathcal{S}_2$ (gráfico de $g_1$) y se cumple que: $\mathcal{S}^+ = \mathcal{S}_1 + \mathcal{S}_2$
y también que:
\[
	\iint\limits_{\mathcal{S}^+}{\vec{R}dS} =
	\iint\limits_{\mathcal{S}_1}{\vec{R}dS} +
	\iint\limits_{\mathcal{S}_2}{\vec{R}dS}
\]
$\mathcal{S}_1$ y $\mathcal{S}_2$ están orientadas con tercera componente negativa y positiva respectivamente\footnote{Si esto no se entiende avisar y se agrega mejor explicación}.\\
\underline{Parametrizaciones} \\ 
\[
	\varphi_1:\varphi_1(x,y) = (x,y,f_1(x,y))
\]
con $(x,y)\in D_1$. \\ 
\[
	(\varphi_1)_x \times (\varphi_1)_y = \left(1, 0, \frac{\partial{f_1}}{\partial{x}}\right) \times\left(0, 1, \frac{\partial{f_1}}{\partial{y}}\right) = \left(-\frac{\partial{f_1}}{\partial{x}},-\frac{\partial{f_1}}{\partial{y}},1\right)
\]
Como dijimos, $\vec{n}$ en $\mathcal{S}_1$ tiene componente negativa por lo que $\varphi_1$ \underline{no} es compatible con la orientación.
Así
\begin{gather*}
	\begin{aligned}
		\iint\limits_{\mathcal{S}_1}{\vec{R}dS} & = - \iint\limits_{D_1}{\vec{R}(\varphi_1(x,y))\cdot ((\varphi_1)_x\times (\varphi_1)_y)~dxdy} \\ 
		& = - \iint\limits_{D_1}{(0,0,R(\varphi_1(x,y)))\cdot (-\frac{\partial{f_1}}{\partial{x}},-\frac{\partial{f_1}}{\partial{y}},1)~dxdy} \\ 
		& = - \iint\limits_{D_1}{R(x,y,f_1(x,y))dxdy}
	\end{aligned}
\end{gather*}
Análogamente, tenemos
\[
	\varphi_2:\varphi_2(x,y) = (x,y,g_1(x,y))
\]
con $(x,y)\in D_1$. \\ 
\[
	(\varphi_2)_x \times (\varphi_2)_y = \left(1, 0, \frac{\partial{g_1}}{\partial{x}}\right) \times\left(0, 1, \frac{\partial{g_1}}{\partial{y}}\right) = \left(-\frac{\partial{g_1}}{\partial{x}},-\frac{\partial{g_1}}{\partial{y}},1\right)
\]
Como dijimos, $\vec{n}$ en $\mathcal{S}_2$ tiene componente positiva por lo que $\varphi_2$ \underline{es} compatible con la orientación.
Así
\begin{gather*}
	\begin{aligned}
		\iint\limits_{\mathcal{S}_2}{\vec{R}dS} & = \iint\limits_{D_1}{\vec{R}(\varphi_2(x,y))\cdot ((\varphi_2)_x\times (\varphi_2)_y)~dxdy} \\ 
		& = \iint\limits_{D_1}{(0,0,R(\varphi_2(x,y)))\cdot (-\frac{\partial{g_1}}{\partial{x}},-\frac{\partial{g_1}}{\partial{y}},1)~dxdy} \\ 
		& = \iint\limits_{D_1}{R(x,y,g_1(x,y))dxdy}
	\end{aligned}
\end{gather*}
Entonces
\[
	\iint\limits_{\mathcal{S}^+}{\vec{R}dS} =
	\iint\limits_{\mathcal{S}_1}{\vec{R}dS} +
	\iint\limits_{\mathcal{S}_2}{\vec{R}dS} =
	- \iint\limits_{D_1}{R(x,y,f_1(x,y))dxdy}~
	+ \iint\limits_{D_1}{R(x,y,g_1(x,y))dxdy}
\]
\[
	\iint\limits_{\mathcal{S}^+}{\vec{R}dS} =
	\iiint\limits_E{R_z dxdydz}
\]
Las otras igualdades son análogas.
\end{proof}


\chapter{Campos y potenciales}
\begin{defbox}{Campo gradiente}
	Sea $\vec{X}:U \subseteq \reals^n \rightarrow \reals^n$ un campo vectorial continuo en un abierto $U$.
	Diremos que $\vec{X}$ es un \textbf{campo gradiente} en $U \iff$ existe una función $f: U \subseteq \reals^n \rightarrow \reals$ de clase $C^1$ tal que
	\begin{equation}
		\vec{X} = \nabla f~~~~~~~~~~~ \text{en }U
		\label{eq:campo_gradiente}
	\end{equation}
	A la función $f$ se le llama \textbf{potencial escalar} del campo $X$ en $U$.
\end{defbox}

\begin{teobox}{Fundamental para campos gradientes}
	Sea $\vec{X}:U \subseteq \reals^n \rightarrow \reals^n$ un campo vectorial continuo en un abierto $U$ y $\mathcal{C}$ una curva regular y simple en $U$ con origen en $A$ extremos en $B$.
	Si $\vec{X}$ es un campo gradiente en $U$ entonces
	\[
		\int_{\mathcal{C}}{\vec{X}ds} = f(B)-f(A)
	\]
	donde $f$ es un potencial escalar del campo $\vec{X}$.
\end{teobox}
\begin{proof}
	Como $\vec{X}$ es un campo gradiente en $U$, existe un potencial escalar $f$ tal que 
	\[
		\vec{X} = \nabla f
	\]
	 en $U$. Luego si consideramos una parametrización $\alpha : [a,b]\rightarrow \reals^3$ de la curva $\mathcal{C}$ con $\alpha (a) = A$ y $\alpha (b) = B$ \\ 
	 Tenemos que:
	 \begin{gather*}
		 \begin{aligned}
			 \int\limits_{\mathcal{C}}{\vec{X}ds} & = \int\limits_{\mathcal{C}}{\nabla f~ ds} \\ 
												  & = \int\limits_a^b{\nabla f(\alpha (t))\cdot \alpha '(t) dt} \\
												  & = \int\limits_a^b{[f(\alpha (t))]' dt}& \text{por regla de la cadena} \\
												  & = f(\alpha (b))-f(\alpha (a))\\
												  & = f(B)-f(A)\\
		 \end{aligned}
	 \end{gather*}
\end{proof}

\begin{defbox}{Campo conservativo}
	Sea $\vec{X}:U \subseteq \reals^3 \rightarrow \reals^3$ un campo vectorial continuo en un abierto $U$.
	Diremos que $\vec{X}$ es un \textbf{campo conservativo} en $U \iff \int_{\mathcal{C}_1}{\vec{X}ds} = \int_{\mathcal{C}_2}{\vec{X}ds}$
	para todo par de curvas $\mathcal{C}_1$ y $\mathcal{C}_2$ contenidas en $U$ que tengan el mismo origen y  el mismo final.
	Esto se llama la propiedad de \textbf{independencia de camino}.
\end{defbox}

\begin{propobox}
	Sea $\vec{X}: U \subseteq \reals^3 \rightarrow \reals^3$ un campo vectorial continuo en un abierto $U$. Entonces
	$\vec{X}$ es un \textbf{campo conservativo} en $U \iff \int\limits_{\mathcal{C}}{\vec{X}ds} = 0$ para cualquier curva $\mathcal{C}$ cerrada y simple contenida en $U$. 
\end{propobox}
\begin{proof}
	$(\Rightarrow )$ Sea $\mathcal{C}$ una curva cerrada y simple contenida en $U$. \\ 
	Consideramos dos puntos cualesquiera $A$ y $B$ en $\mathcal{C}$ e indicamos por $\mathcal{C}_1$ al arco de $A$ a $B$ y $\mathcal{C}_2$ al arco de $B$ a $A$. \\ 
	\[
		\mathcal{C} = \mathcal{C}_1 \cup \mathcal{C}_2
	\]
	Luego
	\begin{gather*}
		\begin{aligned}
			\oint\limits_{\mathcal{C}}{\vec{X}ds} & = \int\limits_{\mathcal{C}_1}{\vec{X}ds} + \int\limits_{\mathcal{C}_2}{\vec{X}ds} \\ 
												  & = \int\limits_{\mathcal{C}_1}{\vec{X}ds} - \int\limits_{-\mathcal{C}_2}{\vec{X}ds} = 0 		
		\end{aligned}
	\end{gather*}
	(como ambas curvas tiene mismo origen y mismo final, por la independencia de camino las integrales son iguales)

	$(\Leftarrow)$ Consideramos un par de curvas $\mathcal{C}_1$ y $\mathcal{C}_2$ contenidas en $U$ que tienen el mismo origen $A$ y el mismo final $B$. \\ 
	Luego, $\mathcal{C} = \mathcal{C}_1 \cup (-\mathcal{C}_2)$ es una curva cerrada y simple en $U$ y por hipótesis
	\begin{gather*}
		\begin{aligned}
			\oint\limits_{\mathcal{C}}{\vec{X}ds} = 0 & \iff \oint\limits_{\mathcal{C}_1 \cup (-\mathcal{C}_2)}{\vec{X}ds} = 0 \\
													  & \iff \int\limits_{\mathcal{C}_1}{\vec{X}ds} + \int\limits_{-\mathcal{C}_2}{\vec{X}ds}  = 0 \\ 
													  & \iff \int\limits_{\mathcal{C}_1}{\vec{X}ds} = - \int\limits_{-\mathcal{C}_2}{\vec{X}ds} \\ 
													  & \iff \int\limits_{\mathcal{C}_1}{\vec{X}ds} =  \int\limits_{\mathcal{C}_2}{\vec{X}ds} \\ 
		\end{aligned}
	\end{gather*}
\end{proof}

\underline{Observación} \\ 
Para curvas cerradas, en física se cambia $\int{}$ por $\oint{}$. En estas notas puede que se utilice esa notación en algun momento.

\begin{defbox}{Campo cerrado}
	Sea $\vec{X}: U \subseteq \reals^n \rightarrow \reals^n$ un campo vectorial de clase $C^1$ en un abierto $U$. \\ 
	Diremos que $\vec{X}$ es \textbf{cerrado} en $U\iff \mathds{J}\vec{X}$ es una matriz simétrica en $U$.\\
	Para el caso de $\reals^2$ y $\reals^3$ tenemos que: \\

	\begin{array}{l|l}
		\reals^2 & \reals^3 \\ 
		\vec{X} = (P,Q) & \vec{X} = (P,Q,R) \\ 
		Q_x - P_y = 0 & Rot(\vec{X}) = 0
	\end{array}
\end{defbox}


% copiamos la clase

\begin{defbox}{Campo rotor}
	Sea $\vec{X}:U \subseteq \reals^3 \rightarrow \reals^3$ un campo vectorial continuo en un abierto $U$.
	Diremos que $\vec{X}$ es un \textbf{campo rotor} en $U \iff$ existe un campo escalar $\vec{F}:U\subseteq \reals^3 \rightarrow \reals^3$ de clase $C^1$ tal que
	\[
		\vec{X} = Rot(\vec{F}) ~~~~~~~~~~~~ \text{en }U
	\]
	Al campo vectorial $\vec{F}$ se le llama \textbf{potencial vector} del campo $\vec{X}$ en $U$.
\end{defbox}

\underline{Observación}
Si $\vec{F}$ es un potencial vector de $\vec{X}$ en $U$.
$\Rightarrow \vec{F} + \nabla g$ también es un potencial vector de $\vec{X}$ en $U$
\[
	Rot\left( \vec{F} + \nabla g \right) = Rot(\vec{F}) + Rot(\nabla g) = Rot(\vec{F})
\]
También  hay una analogía en cuanto a las propiedades entre los campos gradientes que vimos anteriormente y los campos rotores. \\
Sea $\vec{X}$ un campo rotor en $U \Rightarrow$
\[
	\iint_{S}{\vec{X}dS} = \int_{\partial S}{\vec{F}ds}
\]

\end{document}
